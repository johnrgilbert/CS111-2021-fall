\documentclass[11pt]{article} 

\usepackage{amssymb,amsmath}

\newcommand{\numpy}{{\tt numpy}}            % tt font for numpy
\newcommand{\scipy}{{\tt scipy}}            % tt font for scipy
\newcommand{\matplotlib}{{\tt matplotlib}}  % tt font for matplotlib

% \topmargin -1in
% \textheight 9in
% \oddsidemargin  -.25in
% \evensidemargin -.20in
% \textwidth 7in
\topmargin -.5in
\textheight 7.5in
\oddsidemargin  -0.25in
\evensidemargin -0.20in
\textwidth 7in

\begin{document}

$$\mbox{\Large \bf CS 111: Homework 4: Due by 11:59 pm Sunday, October 24, 2021}$$
\par\smallskip\noindent
{\bf Submit your paper as one PDF file,
and tell GradeScope which page(s) each problem is on.
If you worked with a partner, 
you must each separately write up and turn in your own homework paper, 
and report the name of your partner.
No groups of more than two.
}

\par\bigskip
{\bf 1.}
Do problem 2.3 on pages 32--33 of the NCM book, 
showing the {\tt numpy} code you use and its output. 
Note: To understand intuitively what the problem means by 
``assume that joint 1 is rigidly fixed both horizontally and vertically 
and that joint 8 is fixed vertically,'' 
think of the truss as a (2-dimensional) drawbridge across a river, 
with the left end being a hinge and the right end lying on the ground.


\par\bigskip
{\bf 2.} Recall that a symmetric matrix $A$ is {\em positive definite}
(SPD for short) if and only if $x^TAx>0$ for every nonzero vector $x$.

\par\medskip
{\bf 2.1} Find a 2-by-2 matrix $A$ that (1) is symmetric, (2) is not singular,
and (3) has all its elements greater than zero, but (4) is {\em not} SPD.
Show a nonzero vector $x$ such that $x^TAx<0$.

\par\medskip
{\bf 2.2} Let $B$ be an $m$-by-$n$ matrix ($m$ and $n$ may or may not be equal) 
whose rank is $n$.
Prove that the matrix $A=B^TB$ is SPD (mathematically, not experimentally).

\par\bigskip
{\bf 3.} 
If $A$ is symmetric, we don't need to store all $n^2$ of its elements; 
we can just store the $n(n+1)/2$ elements of the upper triangle of $A$, for example.
If $A$ is symmetric and also positive definite then there is a symmetric
version of Gaussian elimination called {\em Cholesky factorization}.
You can read about Cholesky and his factorization in NCM problem~2.5 (pages 35--36),
but don't do that problem. 

The Cholesky factorization of an SPD matrix is
$$A = R^TR,$$
where $R$ is an upper triangular matrix with all its diagonal elements positive.
Notice that there's only one triangular matrix $R$ involved, so computing
the factorization should only need to compute $n(n+1)/2$ numbers, 
not $n^2$ numbers like LU factorization.

One way to get $R$ from $A$ is to factor $A=LU$ with no pivoting 
(there's a theorem that says this is possible, and stable, if $A$ is SPD);
then write $U = DV$ where $D$ is diagonal and $V$ is upper triangular with
ones on the diagonal; then show that $L=V^T$ so that $A=V^TDV$; 
then finally take $R=\sqrt D\, V$, where $\sqrt D$ is just the diagonal matrix
of square roots of diagonal elements of $D$; 
then we have $A=V^TDV=R^TR$ as desired.
However, this method does twice as much work as it needs to, 
because it computes all $n^2$ elements of $L$ and $U$.

Your assignment is to write a routine {\tt R = Cfactor(A)} that returns the 
factor $R$ without ever touching the lower triangle of~$A$ or the lower
triangle of~$R$ (or any other $n$-by-$n$ matrix).
For full credit, your routine should also only do about half as many
arithmetic operations as {\tt L, U = cs111.LUfactorNoPiv(A)}.
For debugging, you can generate a random $n$-by-$n$ SPD matrix $A$ by saying
\begin{verbatim}
    B = np.random.randn(n, n)
    A = B.T @ B
\end{verbatim}
Explain in English (in LaTeX) how your {\tt Cfactor()} works.
Demonstrate that it works by generating a 10-by-10 SPD matrix $A$ as above,
generating a random 10-vector $b$, and comparing the solution to $Ax=b$
from {\tt x = cs111.LUsolve(A,b)} to the solution you get by saying
\begin{verbatim}
    R = Cfactor(A)
    y = cs111.Lsolve(R.T, b)
    x = cs111.Usolve(R, b)
\end{verbatim}
Finally, do an experiment to compare the running time of your {\tt Cfactor(A)}
with that of {\tt LUfactorNoPiv(A)}, for a range of values of $n$ up to large
enough that the routines take a few seconds to run. Report your running times,
and make a plot of the ratio of 
{\tt Cfactor(A)} time to {\tt LUfactorNoPiv(A)} time against $n$.
(You can time one line of code in Jupyter by saying {\tt \%time line-of-code}, 
or you can time a whole window by starting it with {\tt \%\%time}.)

\par\bigskip
{\bf 4.} 
Here you will experiment with solving $Ax=b$ using various solvers
from class and from {\tt numpy}.
For this problem, you should use the 3-D version of the temperature matrix
from {\tt make\_A\_3D()}.
You can use the version of {\tt make\_A\_3D()}
you wrote for Homework 3, or if you prefer you can use 
my version (which is in the latest update of {\tt cs111/temperature.py} on GauchoSpace).
For a right-hand side $b$, use the vector of row sums, {\tt b = A @ np.ones(n)},
so that you know that the exact solution to $Ax=b$ is the vector of all ones.

Experiment with solving $Ax=b$ for the temperature $x$, 
for various values of $k$, using five different solvers as follows.
For each solver, you should report (showing code and output) the
largest value of $k$ for which that solver could solve $Ax=b$ within 30 seconds.
For all but the last solver, use the sparse version of $A$ from {\tt make\_A\_3D()}.
\begin{itemize}
\item The {\tt cs111.CGsolve()} conjugate gradient solver, from class.
(You can vary the arguments {\tt tol} and {\tt max\_iters} to
make it find a more accurate solution.)
\item The {\tt cs111.Jsolve()} Jacobi solver, also from class.
(Again you can vary {\tt tol} and {\tt max\_iters}.)
\item The {\tt scipy} sparse conjugate gradient solver {\tt scipy.sparse.linalg.cg()}.
\item The {\tt scipy} sparse LU solver {\tt scipy.sparse.linalg.spsolve()}.
\item The dense LU solver {\tt cs111.LUsolve()} from class.
For this solver, you will have to convert $A$ to a dense array with {\tt A.toarray()}. 
Warning! This will run out of memory if $k$ gets very big.
\end{itemize}

For each solve, measure and report the run time, the relative residual norm, 
and the relative error norm $||x_{\mbox{exact}} - x||/||x_{\mbox{exact}}||$.
Which solvers are more accurate? Which are faster? 
How do the answers to these questions change as you change $k$?

Warning: Start with very small values of $k$, and be cautious as you increase $k$!
The matrices get big in a hurry.
Different solvers will fall over for different values of $k$.

\end{document}
